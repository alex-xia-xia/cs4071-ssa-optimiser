\documentclass[a4paper,10pt]{report}
\usepackage[utf8]{inputenc}

% Title Page
\title{CS4071 SSA Optimiser}
\author{
Miles McGuire\\
\and
Tom Mason\\
\and
Stephen Rogers\\
\and
Lonan Hugh Lardner
}


\begin{document}
\maketitle

\tableofcontents

\chapter{Introduction}
The aim of this assignment is to design and implement an optimiser for ARM7 assembly. The problem has been split up
into three parts, one of which will be described in this report. Specifically, we will examine the conversion of an
intermediate representation (IR) into SSA (Static Single Assignment) form, various optimisations performed on the IR 
in SSA form, and conversion back out of this form.

In the Design chapter, we will begin by describing the IR that is passed into our application to be optimised. Then we 
will examine how conversion into SSA form is achieved. In the following section, we will describe the various 
optimisations that are performed on the IR in SSA form. Finally, we will look at how conversion out of SSA is completed.

In the Implementation chapter, we will give a brief overview of how the application is implemented, focusing on code 
structure and any technologies used.

\chapter{Design}

\section{Intermediate Representation}
The intermediate representation was designed specifically to use JSON for passing between different parts of the overall 
application. JSON was chosen because of its simplicity and the wide range of libraries available for working with it.

Any program that is being passed into our optimiser is represented by a single object. This object will generally contain
two members. The first is a comments section, which can be ignored by the optimiser. It exists primarily for human reading 
only, to place context on the program represented. The second is an array of objects representing the basic blocks.

Each of these basic block objects contain three members. The first is the name of the basic block. The second is an array
of objects, each representing an instruction contained by the basic block. The third and final member is an array which
contains the name of all blocks that succeed this one. Each object representing an instruction must contain an ``op'' 
member. They may also contain members for a destination variable and up to two source variables.

\section{Conversion into SSA}
Before conversion into SSA form can begin, we need to generate certain pieces of information about the program being converted.
First, we must generate the Control Flow Graph (CFG) for the program. This is done simply in two steps. We iterate over
the program, adding each basic block as a node in the graph. Then we iterate over the program again, adding an edge from a node
x to a node y where y is in the ``next block'' set for node x.

From the CFG we can now calculate the Dominance Frontier (DF) of each of the nodes in the graph (i.e. each basic block).
The DF of each node is needed to determine where to insert \(\phi\)-functions later in the process. The DF of each node x is 
the set {w} where x dominates some immediate predecessor of w but does not strictly dominate w. The set of all DFs can be 
calculated programmatically by iterating over the CFG.

Finally, we calculate the set of ``definition sites'' for each variable in the program. This set, for a variable contains 
all basic blocks in which there is an assignment to that variable.

Once we have all of this information, we can begin inserting \(\phi\)-functions into the program. This is done using the
following algorithm.

\begin{verbatim}
for each variable x in the program
    worklist = definition sites of x

    while worklist is not empty
        remove node n from the worklist

        for each node y in DF(n)
            if x does not have a phi-function in y
                insert phi-function for x at the top of block y

                if y is not in the definition sites for x
                    add y to worklist
\end{verbatim}

Once we have successfully inserted all \(\phi\)-functions, we can begin renaming variables to complete the conversion into
SSA. 

A top-down, depth-first approach is taken to renaming variables. Starting with the entry block to the graph, variables are renamed as they
are encountered. The algorithm used renames variables on a block by block basis. It renames all variables in the current block
and then renames all the relevant sources to \(\phi\)-functions in all successors. Once this is done, the algorithm moves onto
the next block in the graph, taken from the current block's list of successors.

\section{Optimisations}

\subsection{Dead-code elimination}
Simple dead-code elimination is achieved using the algorithm examined in class on the ``SSA Optimization Algorithms'' handout.

\begin{verbatim}
worklist = all variables in the SSA program
while worklist is not empty
    remove variable v from worklist
    if v's list of uses is empty
        let s be v's statement of definition
        if s has no side effects (other than assignment to v)
            delete s from the program
            for each variable x used by s
                delete s from the list of uses for x
                worklist = worklist U {x}
\end{verbatim}
This algorithm simply removes any assignment statements to variables which are not used at any other point in the program.
Whenever a statement is removed, the list of uses of any variables used by that statement must be updated. When this is done
those variables are added back onto the worklist to be checked again. This continues until the worklist is empty.

\subsection{Simple constant propagation}
The algorithm used for simple constant propagation is also taken from the ``SSA Optimization Algorithms'' handout.
The algorithm has been modified slightly to include copy propagation and constant folding into the one process.

\begin{verbatim}
worklist = all statements in the SSA program
while worklist is not empty
    remove statement s from worklist
    if s is a phi-function with all the same arguments
        replace s with a copy operation
    if s is a foldable operation
        if both source operands are constants
            fold constant statement s
    if s is a copy of some constant c to destination v
        delete s from the program
        for each statement t that uses v
            substitute c for v in t
            worklist = worklist U {t}
\end{verbatim}
This algorithm performs three distinct transformations on the original SSA program.
\begin{itemize}
 \item Eliminates \(\phi\)-functions where all \(\phi\)-function operands are equal, replacing such functions with a copy operation.
 \item Constant folds operations on two constant variables, by performing the specified operation and replacing the statement
        with a copy operation with the result, eg. ``ADD R0, \#1, \#5'' becomes ``MOV R0, \#6''.
 \item Copy propagation taking single argument \(\phi\)-functions or copy assignments of the form x = \(\phi\)(y) or x = y, 
        deleting them and replacing all uses of `x` by `y`.
\end{itemize}

\subsection{Aggressive dead code elimination}
This optimisation aggressively finds and eliminates dead code using the algorithm described in the ``SSA Optimization 
Algorithms'' handout. The algorithm performs the following 9 steps.
\begin{enumerate}
\item Marks all statements to be deleted.
\item Unmarks statements that perform "live operations":
  \begin{itemize}
  \item I/O
  \item Memory writes (STR)
  \item Branch \& Exchange (BX) and Branch \& Link (BL)
  \item Software Interrupts (SWI)
  \item Status register updates (CMP), operations with the `S` flag.
  \end{itemize}
\item Unmarks statements defining variables used in live statements.
\item Unmarks conditional branches that directly control execution of live statements.
\item Removes blocks that will not be reachable from the START node after deletion.
\item Deletes all marked statements.
\item Removes variables from  \(\phi\)-functions whose definitions have been eliminated.
\item Removes blocks which contain no statements.
\item Iteratively finds a least fixed point at which no further statements are removed.
\end{enumerate}

\subsection{Conditional constant propagation}
Once again, the algorithm used for conditional constant propagation is described in the ``SSA Optimization Algorithms'' handout.
At the beginning, this algorithm assumes that a basic block cannot be executed until there is evidence that shows it can be.
Similarly, the algorithm assumes that a variable is never assigned to, unless there is evidence that shows it is.

In the algorithm, the set V tracks the ``evidence'' associated with each variable in the program. For any variable v,
V[v] is true if there is evidence that v will have at least two values at various points in the program, or an unpredictable
value (e.g. read from a file). V[v] is false otherwise. Also, the set D tracks whether or not basic blocks should be deleted or 
not. At the start, D[b] is true for all blocks b in the program. If it is found that a block is reachable, then D[b] is set to false.
Whenever a block is found to be reachable, it is added to the worklist.

The algorithm works by iterating over a worklist of basic blocks. This worklist initially contains only the starting block,
since this is the only block at this point that we can say for certain will be executed. Blocks are popped off the worklist
one at a time. Whenever a block with a single successor s is encountered, D[s] is set to false. In the event that there are two
successors to a block (called B1 and B2), if both source operands to the conditional jump are found to be constant, then the result 
of the jump can be evaluated at compile time. Depending on the result, either D[B1] is set to false or D[B2] is set to false.

The algorithm performs all of the steps outlined in the handout iteratively on each statement of each basic block on the worklist.
Once the worklist is empty, the algorithm deletes any basic blocks b where D[b] is still set to true.

\section{Conversion out of SSA}
In order to convert out of SSA form, we used the algorithm described by Sreedhar et al. examined in class. Specifically, we used 
the description of the algorithm outlined in the handout of Chapter 5 of ``Optimization in Static Single Assignment Form'' from 
the Sassa Laboratory in Tokyo Institute of Technology.

In broad terms, this algorithm involves converting the program into CSSA (Conventional SSA) form and then
coalescing \(\phi\)-functions to take the program out of SSA form altogether.

Conversion into CSSA form is done using the naive translation method outlined in the handout (i.e. Method I on page 56).
The following is basic pseudocode outlining the overall algorithm used.

\begin{verbatim}
for each block b in the program
    for each phi-function p in b
        for each source s of p
            Add a copy instruction from s into temporary variable 
                ts at the end of the predecessor block
            Replace s in p with ts
        Replace destination d of p with temporary variable td
        Insert a copy instruction from td to d immediately after p
\end{verbatim}

Once this has completed, we can begin coalescing \(\phi\)-functions and ultimately remove them. This is done by first combining
all \(\phi\)-functions that share one or more variables, either as sources or destination. A simple iterative approach over all 
\(\phi\)-functions in the program is used to do this combining. Next, assign a single variable name to all variables in each 
\(\phi\)-function (i.e. one name per \(\phi\)-function). 

Now, we can replace all uses of the variables in the \(\phi\)-functions with their newly assigned names. This is done by iterating
over the entire program and replacing names where necessary. Finally, the \(\phi\)-functions can be removed. This brings the program
fully out of SSA form.

\chapter{Implementation}
Everything outlined in the above Design section has been implemented in Python 2.7. A simple modular design was taken for the
implementation. Each activity outlined in the above section is separated into its own Python file. This means that optimisations
can be added and removed by a user, based upon their own needs and requirements. The following is a list of source files that
are apart of our application, with a brief description of what they do.

\begin{itemize}
\item \textbf{graphs.py} - Contains a graph class which is used to represent a Control Flow Graph. This class also contains all relevant
                  information to the optimisations performed in the other files (for example the dominance tree and dominance
                  frontiers of each node).
                  
\item \textbf{util.py} - Includes simple functions that are common to several optimisations.

\item \textbf{ssa.py} - Contains all the functionality described for converting a program into SSA form. Conversion can be invoked via a
               top level function \textit{toSSA(code)} which will insert \(\phi\)-functions and rename all variables.
               
\item \textbf{dead\_code\_elimination.py} - Responsible for performing the dead-code elimination algorithm outlined in the above section.

\item \textbf{constant\_propagation.py} - Responsible for performing the simple constant propagation algorithm outlined above.

\item \textbf{aggressive\_dead\_code\_elimination.py} - Responsible for performing the aggressive dead code elimination algorithm outlined above.

\item \textbf{conditional\_constant\_propagation.py} - Responsible for performing the conditional constant propagation algorithm outlined above.

\item \textbf{fromSSA.py} - Includes all the functionality outlined for conversion out of SSA form. This conversion can be invoked using the
                   function \textit{fromSSA(code)}.

\item \textbf{\_\_init\_\_.py} - Contains a single function \textit{optimise(code)} which brings together all of the files described here. This
                    function will convert the given code into SSA form, perform the 4 optimisations described and finally convert it
                    back out of SSA form.
\end{itemize}

\chapter{Conclusion}
In this report, we have examined the design and implementation of an optimiser for ARM7 assembly code. This optimiser is capable of
converting a given ARM7 program (pre-formatted in the JSON-based IR outlined) into SSA form, performing 4 distinct optimisations on
this new form and finally converting back out of SSA form into normal form. 

\end{document}          
